
\documentclass{wit_syllabus}

% Professor/Course information
\author{Nate Derbinsky}
\date{Fall 2015}
\course{Engineering and Technology}{COMP128}{Computer Science I}

\begin{document}

% Create logo/title
\maketitle

% Create info box
\begin{SyllabusHeader}
\instructor
\office{Dobbs 140}{M 12-1 and by appointment}
\contact{(617) 989-4287}{derbinskyn@wit.edu}{http://derbinsky.info}
\topic{Credits/Hours}{3/2/4}
\end{SyllabusHeader}

% Description
\SyllabusSection{Course Description}
Insert course description here.  Should match the course description that appears in the Catalog.

% Prerequisites
\SyllabusSection{Course Prerequisites/Corequisites}
Insert any course pre/corequisites here.  This should match the course description that appears in the Catalog. If there are no prerequisites or corequisites please note this.

% Required books
\begin{SyllabusBooks}{Required Textbook(s)}
\bookInfo{Savitch, Walter}{Java: Introduction to Problem Solving and Programming}{7th ed}{Addison-Wesley}{2014}{978-0133766264}
\end{SyllabusBooks}

% Bookstore
\SyllabusBookstore

% Recommended books
\begin{SyllabusBooks}{Recommended Learning Materials}
\book{\textit{Insert other helpful resources should be listed here.}}
\end{SyllabusBooks}

% Learning outcomes
\SyllabusSection{Course Learning Outcomes}

At the completion of this course, the student should be able to:
\begin{itemize}
\item Foo
\item Bar
\item Baz
\item Quz
\end{itemize}

% Instructional methodologies
\SyllabusSection{Instructional Methodologies}
This course will combine traditional lecturing with hands-on assignments that reinforce the lecture material.  
In particular, lectures will focus on concepts and ideas while the assignments will provide concrete experience and skills.

% Attendance
\SyllabusSection{Attendance Policy}
Students are expected to attend classes regularly, take tests, and submit papers and other work at the times specified by the instructor. 
Students who are absent repeatedly from class or studio will be evaluated by faculty responsible for the course to ascertain their ability to achieve the course objectives and to continue in the course.  
Instructors may include, as part of the semester's grades, marks for the quality and quantity of the student's participation in class.  
At the discretion of the instructor, a student who misses 15 percent of class may be withdrawn from the course by the instructor. 
A grade of \texttt{WA} will appear on the student's official transcript as a result.

% Grading
\SyllabusSection{Grading Policy}
Insert grading policy here.  Your policy must state:
\begin{itemize}
\item specific assignments a student must complete to meet the learning outcomes.
\item number of assignments in each category that are required.
\item relative weight of each assignment
\item if the assignment is a project, presentation, paper, etc., criteria must be established so that students will understand exactly how they will be graded (may be handed out to students under separate cover).
\end{itemize}

% Grading system
\SyllabusGradingSystem

% Add/Drop
\SyllabusDropAdd

% Makeup

\SyllabusSection{Make-Up Policy}

All assignments have a specific due date and time.  
Submissions will be accepted after the deadline with a 50\% penalty.  
The assignment will be graded and returned as normal, but the grade will be recorded as half of what was earned.  
For example, an on-time submission might receive a grade of 90 points.  
The same assignment submitted after the deadline would receive 45 points ($90 \times 0.5$).  
\\

Students who miss scheduled exams will not, as a matter of course, be able to make up those exams.  
If there is a legitimate reason why a student will not be able to complete an assignment on time or not be present for an exam, then they should contact the instructor beforehand.  
Under extreme circumstances, as decided on a case-by-case basis by the instructor, students may be allowed to make up assignments or exams without first informing the instructor.

% Academic Support
\SyllabusAcademicSupport

% Academic Honesty
\SyllabusAcademicHonesty

% Student Accountability
\SyllabusSection{Student Accountability Statement}

Behavior unbecoming a student is any violation of a published Wentworth policy in an academic environment, and/or any behavior that individual faculty or staff determines is unacceptable in his or her classroom, laboratory, or other academic area or function. 
Behavior unbecoming a student in an academic environment will not be tolerated. 
Violations of behavioral expectations may be forwarded to the Office of Community Standards for disciplinary action.
\\

Wentworth takes violations of academic dishonesty and misconduct very seriously. 
Sanctions for such violations include, but are not limited to, a grade of ``F'', removal from a course, Institute suspension, or Institute expulsion.

% Disability
\SyllabusDisability

% COF
\SyllabusCOF

% Schedule

\SyllabusSection{WeeklySchedule}

It will benefit you greatly to complete the assigned reading \textit{before} attending the lecture.
\\

\begin{SyllabusSchedule}
\week{Introduction to Computation and Programming}{1.1, 1.2, 1.3}{}
\week{Variables, I/O, Types, Strings}{2.1, 2.2, 2.3, 2.4}{HW0 due}
\week{Control Flow, Conditionals}{3.1, 3.2, 3.3}{HW1 due}
\week{Exam 1 Review, Expressions, Loops}{4.1, 4.2}{Exam 1, HW2 due}
\week{Loops cont'd}{4.1, 4.2}{HW3 due}
\week{Functions}{5.1, 6.4}{HW4 due}
\week{Exam 2 Review, Arrays}{7.1, 7.2, 7.3}{Exam 2, HW5 due}
\week{Arrays cont'd}{7.1, 7.2, 7.3}{HW6 due}
\week{Testing and Debugging}{1.3}{HW7 due}
\week{Exam 3 Review, Object Oriented Programming}{5.1, 5.2, 5.3, 6.1, 6.7}{Exam 3}
\week{Designing Classes}{5.1, 5.2, 5.3, 6.1, 6.7}{HW8 due}
\week{Exceptions, File I/O}{9.1, 10.1, 10.2}{HW9 due}
\week{Lists}{12.1}{HW10 due}
\week{Advanced Topics (e.g. GUI Programming)}{13}{HW11 due}
\week{Final Exam Review}{}{Final Exam}
\end{SyllabusSchedule}


\end{document}
