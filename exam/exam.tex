
\documentclass{wit_exam}

%% Optional inclusion of packages and/or commands/definitions
\usepackage{listings}
\usepackage{color}
\usepackage{enumerate}

\definecolor{dkgreen}{rgb}{0,0.6,0}
\definecolor{gray}{rgb}{0.5,0.5,0.5}
\definecolor{mauve}{rgb}{0.58,0,0.82}

\lstset{
  %frame=tb,
  language=C++,
  %aboveskip=3mm,
  %belowskip=3mm,
  showstringspaces=false, % shows string space character
  %columns=flexible,
  basicstyle={\small\ttfamily},
  % numbers=left,
  numberstyle=\tiny\color{gray},
  keywordstyle=\color{blue},
  commentstyle=\color{dkgreen},
  stringstyle=\color{mauve},
  breaklines=true,
  breakatwhitespace=true,
  tabsize=3,
  escapeinside=`',
}


% Course/Professor information
\title{Midterm Exam}
\author{Professor Foo}
\date{Summer 1956}
\course{CLS101}{Introduction to Stuffs}


% List of front-page instructions
\ExamInstruction{Don't cheat!}
\ExamInstruction{Be considerate of others}
\ExamInstruction{All responses must be written in \textbf{pen}.}


% Each problem: Name, Points
\ProblemDeclaration{Write a Function \#1}{70}
\ProblemDeclaration{Write a Program}{70}
\ProblemDeclaration{Write a Function \#2}{60}
\ProblemDeclaration{Predict the Output}{60}
\ProblemDeclaration{Fill in the Blanks}{40}
\BonusDeclaration{Bonus}{30}


% Could be anything, but most likely just a sequence of \ProblemDefinition environments
\begin{document}

\begin{ProblemDefinition}[%
Write a function definition that implements the following description and nothing else (you do not need to write the \texttt{main()} function).
]

The function \texttt{foo} must output the string ``foo'' to the terminal (without the quotes).

\end{ProblemDefinition}

%%

\begin{ProblemDefinition}

The program first gets a single value from the user: the mass of an object.
The program then computes and outputs the energy equivalence, as measured by the following seminal equation:
\\

$E=mc^2$
\\

where the constant $c$ is the speed of light (299,792,458 meters per second).
You may use the next page as scratch work -- \textbf{only work on the page will be graded}.

\newpage

\textit{This page intentionally left blank.}

\end{ProblemDefinition}

%%

\begin{ProblemDefinition}[Write a function definition that implements the following description and nothing else (you do not need to write the \texttt{main()} function).]

The function \texttt{bar\_file} is passed an \texttt{ofstream} argument that has already been opened on a file.
The function should output the string ``bar'' to this file (without the quotes).

\end{ProblemDefinition}

%%

\begin{ProblemDefinition}[%
Predict the output to the terminal window when the following program fragments are executed.
]

\begin{tabular*}{1.0\textwidth}{ | p{\dimexpr 0.55\linewidth-2\tabcolsep} | p{\dimexpr 0.45\linewidth-2\tabcolsep} | }
\hline
\textbf{Fragment} & \textbf{Output Seen in Terminal Window} \tabularnewline \hline

\begin{lstlisting}

cout << "foo" << endl;



\end{lstlisting}
& \tabularnewline \hline

\begin{lstlisting}

cout << "bar" << endl;



\end{lstlisting}
& \tabularnewline \hline

\begin{lstlisting}

cout << "baz" << endl;



\end{lstlisting}
& \tabularnewline \hline

\begin{lstlisting}

cout << "quz" << endl;



\end{lstlisting}
& \tabularnewline \hline

\end{tabular*}

\end{ProblemDefinition}

%%

\begin{ProblemDefinition}[%
Fill in the 5 blanks in the program below.
]

\vspace{-0.7cm}
\begin{lstlisting}

#include `\BlankText{ < i o s t r e a m > }'
using namespace `\BlankText{ s t d; }';

// Outputs "hello world" to the terminal
`\BlankText{ i n t }' main()
{
  `\BlankText{ c o u t }' << "hello world" << endl;
  `\BlankText{ r e t u r n }' 0;
}

\end{lstlisting}

\end{ProblemDefinition}

%%

\begin{ProblemDefinition}[%
Each correct response below is worth 10 bonus points.
]

\begin{enumerate}[a)]

\vspace{0.0cm}
\item{%
What\dots \ is your name?
}

\vspace{5.0cm}
\item{%
What\dots \ is your quest?
}

\vspace{5.0cm}
\item{%
What\dots \ is the air-speed velocity of an unladen swallow?
}

\end{enumerate}

\end{ProblemDefinition}

%%

\end{document}
